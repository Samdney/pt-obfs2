% ================================================================================
\documentclass[sigconf, screen]{acmart}

\usepackage{booktabs} % For formal tables
\usepackage{graphicx}
\usepackage{comment}
\usepackage{url}
\usepackage{hyperref}

\usepackage{listings}
\usepackage[n, advantage, operators, sets, adversary, landau, probability, notions, logic, ff, mm, primitives, events, complexity, asymptotics, keys]{cryptocode}

% Copyright
\setcopyright{none}
%\setcopyright{acmcopyright}
%\setcopyright{acmlicensed}
%\setcopyright{rightsretained}
%\setcopyright{usgov}
%\setcopyright{usgovmixed}
%\setcopyright{cagov}
%\setcopyright{cagovmixed}

% ================================================================================
% DOI
%\acmDOI{10.475/123_4}
% ================================================================================
%Conference
%\acmConference[WOODSTOCK'97]{ACM Woodstock conference}{July 1997}{El Paso, Texas USA}
%\acmYear{1997}
%\copyrightyear{2016}

%\acmArticle{4}
%\acmPrice{15.00}
% ================================================================================
% These commands are optional
%\acmBooktitle{Transactions of the ACM Woodstock conference}
%\editor{ABC}
% ================================================================================
\begin{document}
% ================================================================================
\title{Obfs2}	% TODO
% ================================================================================
%\titlenote{Produces the permission block, and copyright information}
\subtitle{Pluggable Transports documentation series.}	% TODO
\subtitlenote{The author believes in the importance of the independence of research and is funded by the public community. If you also believe in this values, you can find ways for supporting the author's work here: \url{https://research.carolin-zoebelein.de/crowdfunding.html}}
% ================================================================================
\author{Carolin Z\"obelein}
\authornote{\url{https://research.carolin-zoebelein.de}, \textit{E-mail address:} contact@carolin-zoebelein.de, PGP: D4A7 35E8 D47F 801F 2CF6 2BA7 927A FD3C DE47 E13B}
\affiliation[obeypunctuation=true]{
	\institution{Independent mathematical scientist}\\
	\streetaddress{Josephsplatz 8},
	\postcode{90403}
  	\city{N\"urnberg},
  	\country{Germany}  	
}
% ================================================================================
\begin{abstract}	% TODO
STATUS: Draft
%This paper provides a sample of a \LaTeX\ document which conforms,
%somewhat loosely, to the formatting guidelines for
%ACM SIG Proceedings.\footnote{This is an abstract footnote}
\end{abstract}
% ================================================================================
%
% The code below should be generated by the tool at
% http://dl.acm.org/ccs.cfm
% Please copy and paste the code instead of the example below.
%
\begin{comment}	% TODO
\begin{CCSXML}
<ccs2012>
 <concept>
  <concept_id>10010520.10010553.10010562</concept_id>
  <concept_desc>Computer systems organization~Embedded systems</concept_desc>
  <concept_significance>500</concept_significance>
 </concept>
 <concept>
  <concept_id>10010520.10010575.10010755</concept_id>
  <concept_desc>Computer systems organization~Redundancy</concept_desc>
  <concept_significance>300</concept_significance>
 </concept>
 <concept>
  <concept_id>10010520.10010553.10010554</concept_id>
  <concept_desc>Computer systems organization~Robotics</concept_desc>
  <concept_significance>100</concept_significance>
 </concept>
 <concept>
  <concept_id>10003033.10003083.10003095</concept_id>
  <concept_desc>Networks~Network reliability</concept_desc>
  <concept_significance>100</concept_significance>
 </concept>
</ccs2012>
\end{CCSXML}

\ccsdesc[500]{Computer systems organization~Embedded systems}
\ccsdesc[300]{Computer systems organization~Redundancy}
\ccsdesc{Computer systems organization~Robotics}
\ccsdesc[100]{Networks~Network reliability}
\end{comment}

\keywords{Tor, Bridge, Scary, Obscuration, Censorship, Circumvention, Pluggable Transport}	% TODO
% ================================================================================
\maketitle
% ================================================================================
\begin{comment}
TODO: Roadmap
=> +: Introduction (only a few lines)
=> -: Spec
=> -: Thread model
=> -: obfsproxy
=> -: "protocol" diagram
=> -: Sample/Wireshark stuff
=> -: "Fun Time"
=> -: "Play Time"
=> -: Strengths and weaknesses
=> -: Prospect
=> -: Conclusion
=> -: Appendix
=> -: What else? ...
\end{comment}
% ================================================================================
\section*{Preamble}
\label{s:preamble}
% --------------------------------------------------------------------------------
This paper is part of a paper documentation series about Pluggable Transports (PTs) \cite{TorPluggableTransports}, how they work and their strengths and weaknesses.
% --------------------------------------------------------------------------------
\section{Introduction}
\label{s:introduction}
% --------------------------------------------------------------------------------
During the history of digital networks, we have been confronted more and more with the phenomenon of internet censorship and blocking by governments \cite{foci12-winter}. So, over the years, more and more circumvention tools were developed and also have been blocked due to deep packet inspections and the detailed analysis of its content. This lead us to, so-called, \textit{Pluggable Transports (PTs)} \cite{TorPluggableTransports}, which help to bypass censorship attemps by transforming the traffic between client and server, in such a ways that it looks like innocent traffic.

In this paper, we will talk about \textit{obfs2}, a protocol obfuscation layer for TCP protocols. We will show how it works, what we can do with it and which strengths and weaknesses it has.
% ---------------------------------------
\subsection{Outline}
\label{ss:outline}
% ---------------------------------------
% TODO
TODO
% ================================================================================
\section{Obfs2}
\label{s:obfs2}
% --------------------------------------------------------------------------------
% TODO: AUTHORS!!!, HISTORY!!!
\textit{Obfs2} is a protocol obfuscation layer for TCP protocols, to keep a third party from telling what protocol is in use based on message contents \cite{TorGitWebObfs2Specification}. It's the continuation of brl's ssh obfuscation protocol \cite{TorGitWebObfs2Specification} \cite{GitHubBrlObfuscatedOpenssh}.
% ---------------------------------------
\subsection{Overview}
\label{ss:overview}
% ---------------------------------------
The protocol consists of two phases.
\begin{itemize}
    \item First: The parties establish keys
    \item Second: The parties exchange superenciphered traffic.
\end{itemize}
% ---------------------------------------
\subsection{Notation}
\label{ss:notation}
% ---------------------------------------
Given are two parties: the 'initiator' (\lstinline[language=C]{INIT}), which opens the connection and can mostly be associated with a client, and the 'responder' (\lstinline[language=C]{RESP}), which accepts the connection and can mostly be associated with a server.

We use the following primitives,
\begin{itemize}
    \item $\mathrm{H\left(x\right)}$ is SHA256 of x
    \item $\mathrm{H^{n}\left(x\right)}$ is $\mathrm{H\left(x\right)}$ called iteratively n times
    \item E(K,s) is the AES-CTR-128 encryption of s using K as key
\end{itemize}

notation
\begin{itemize}
    \item x | y is the concatenation of x and y
    \item UINT32(n) is the 4 byte value of n in big-endian (network) order
    \item SR(n) is n bytes of strong random data
    \item WR(n) is n bytes of weaker random data
    \item "xyz" is the ASCII characters 'x', 'y', and 'z', not NUL-terminated
    \item s[:n] is the first n bytes of s
    \item s[n:] is the last n bytes of s
\end{itemize}

and constants
\begin{itemize}
    \item \lstinline[language=C]{MAGIC_VALUE = 0x2BF5CA7E}
    \item \lstinline[language=C]{SEED_LENGTH = 16}
    \item \lstinline[language=C]{MAX_PADDING = 8192}
    \item \lstinline[language=C]{HASH_ITERATIONS = 100000}
\end{itemize}

\begin{itemize}
    \item KEYLEN (16) is the length of the key used by E(K,s)
    \item IVLEN (16) is the length of the IV used by E(K,s)
    \item HASHLEN (32) is the length of the output of H()
    \item MAC(s, x) = H(s | x | s)
\end{itemize}

according to \cite{TorGitWebObfs2Specification}. A "byte" is an 8-bit octet and we require that HASHLEN >= KEYLEN + IVLEN.
% ================================================================================
\section{Conclusions}
\label{s:conclusions}
% --------------------------------------------------------------------------------
% TODO
% ================================================================================
\appendix
\section{Appendix}
\label{s:appendix}
% --------------------------------------------------------------------------------
% TODO
% ================================================================================
% TODO: entry types and author information, bibliographystyle
%\nocite{*}		% TODO
\bibliography{pt-obfs2}
\bibliographystyle{ACM-Reference-Format}
% ================================================================================
\section*{License}
\label{s:license}
% --------------------------------------------------------------------------------
\begin{center}
	\includegraphics{by-nc-nd.png} \\
	\url{https://creativecommons.org/licenses/by-nc-nd/4.0/}
\end{center}
% ================================================================================
\end{document}
% ================================================================================
