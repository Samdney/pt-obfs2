% ================================================================================
\documentclass[sigconf, screen]{acmart}

\usepackage{booktabs} % For formal tables
\usepackage{graphicx}
\usepackage{comment}
\usepackage{url}
\usepackage{hyperref}

\usepackage{listings}
\usepackage[n, advantage, operators, sets, adversary, landau, probability, notions, logic, ff, mm, primitives, events, complexity, asymptotics, keys]{cryptocode}

% Copyright
\setcopyright{none}
%\setcopyright{acmcopyright}
%\setcopyright{acmlicensed}
%\setcopyright{rightsretained}
%\setcopyright{usgov}
%\setcopyright{usgovmixed}
%\setcopyright{cagov}
%\setcopyright{cagovmixed}

% ================================================================================
% DOI
%\acmDOI{10.475/123_4}
% ================================================================================
%Conference
%\acmConference[WOODSTOCK'97]{ACM Woodstock conference}{July 1997}{El Paso, Texas USA}
%\acmYear{1997}
%\copyrightyear{2016}

%\acmArticle{4}
%\acmPrice{15.00}
% ================================================================================
% These commands are optional
%\acmBooktitle{Transactions of the ACM Woodstock conference}
%\editor{ABC}
% ================================================================================
\begin{document}
% ================================================================================
\title{Obfs2}	% TODO
% ================================================================================
%\titlenote{Produces the permission block, and copyright information}
\subtitle{Pluggable Transport documentation series.}	% TODO
\subtitlenote{The author believes in the importance of the independence of research and is funded by the public community. If you also believe in this values, you can find ways for supporting the author's work here: \url{https://research.carolin-zoebelein.de/crowdfunding.html}}
% ================================================================================
\author{Carolin Z\"obelein}
\authornote{\url{https://research.carolin-zoebelein.de}, \textit{E-mail address:} contact@carolin-zoebelein.de, PGP: D4A7 35E8 D47F 801F 2CF6 2BA7 927A FD3C DE47 E13B}
\affiliation[obeypunctuation=true]{
	\institution{Independent mathematical scientist}\\
	\streetaddress{Josephsplatz 8},
	\postcode{90403}
  	\city{N\"urnberg},
  	\country{Germany}  	
}
% ================================================================================
\begin{abstract}	% TODO
STATUS: Draft
%This paper provides a sample of a \LaTeX\ document which conforms,
%somewhat loosely, to the formatting guidelines for
%ACM SIG Proceedings.\footnote{This is an abstract footnote}
\end{abstract}
% ================================================================================
%
% The code below should be generated by the tool at
% http://dl.acm.org/ccs.cfm
% Please copy and paste the code instead of the example below.
%
\begin{comment}	% TODO
\begin{CCSXML}
<ccs2012>
 <concept>
  <concept_id>10010520.10010553.10010562</concept_id>
  <concept_desc>Computer systems organization~Embedded systems</concept_desc>
  <concept_significance>500</concept_significance>
 </concept>
 <concept>
  <concept_id>10010520.10010575.10010755</concept_id>
  <concept_desc>Computer systems organization~Redundancy</concept_desc>
  <concept_significance>300</concept_significance>
 </concept>
 <concept>
  <concept_id>10010520.10010553.10010554</concept_id>
  <concept_desc>Computer systems organization~Robotics</concept_desc>
  <concept_significance>100</concept_significance>
 </concept>
 <concept>
  <concept_id>10003033.10003083.10003095</concept_id>
  <concept_desc>Networks~Network reliability</concept_desc>
  <concept_significance>100</concept_significance>
 </concept>
</ccs2012>
\end{CCSXML}

\ccsdesc[500]{Computer systems organization~Embedded systems}
\ccsdesc[300]{Computer systems organization~Redundancy}
\ccsdesc{Computer systems organization~Robotics}
\ccsdesc[100]{Networks~Network reliability}
\end{comment}

\keywords{Tor, Bridge, Scary, Obscuration, Censorship, Circumvention, Pluggable Transport}	% TODO
% ================================================================================
\maketitle
% ================================================================================
\begin{comment}
TODO: Roadmap
=> +: Introduction (only a few lines)
=> -: Spec
=> -: Thread model
=> -: obfsproxy
=> -: "protocol" diagram
=> -: Sample/Wireshark stuff
=> -: "Fun Time"
=> -: "Play Time"
=> -: Strengths and weaknesses
=> -: Prospect
=> -: Conclusion
=> -: Appendix
=> -: What else? ...
\end{comment}
% ================================================================================
\section*{Preamble}
\label{s:preamble}
% --------------------------------------------------------------------------------
This paper is part of a paper documentation series about Pluggable Transports (PTs) \cite{TorPluggableTransports}, how they work and their strengths and weaknesses.
% --------------------------------------------------------------------------------
\section{Introduction}
\label{s:introduction}
% --------------------------------------------------------------------------------
During the history of digital networks, we have been confronted more and more with the phenomenon of internet censorship and blocking by governments \cite{foci12-winter}. So, over the years, more and more circumvention tools were developed and also have been blocked due to deep packet inspections and the detailed analysis of its content. This lead us to, so-called, \textit{Pluggable Transports (PTs)} \cite{TorPluggableTransports}, which help to bypass censorship attemps by transforming the traffic between client and server, in such a ways that it looks like innocent traffic.

In this paper, we will talk about \textit{obfs2}, a protocol obfuscation layer for TCP protocols. We will show how it works, what we can do with it and which strengths and weaknesses it has.
% ---------------------------------------
\subsection{Outline}
\label{ss:outline}
% ---------------------------------------
% TODO
TODO
% ================================================================================
\section{Obfs2}
\label{s:obfs2}
% --------------------------------------------------------------------------------
% TODO: AUTHORS!!!, HISTORY!!!
\textit{Obfs2} is a protocol obfuscation layer for TCP protocols, to keep a third party from telling what protocol is in use based on message contents \cite{TorGitWebObfs2Specification}. It's the continuation of brl's ssh obfuscation protocol \cite{TorGitWebObfs2Specification} \cite{GitHubBrlObfuscatedOpenssh}.
% ---------------------------------------
\subsection{Overview}
\label{ss:overview}
% ---------------------------------------
The protocol consists of two phases.
\begin{itemize}
    \item First: The parties establish keys
    \item Second: The parties exchange superenciphered traffic.
\end{itemize}
% ---------------------------------------
\subsection{Notation}
\label{ss:notation}
% ---------------------------------------
Given are two parties: the 'initiator' (\lstinline[language=C]{INIT}), which opens the connection and can mostly be associated with a client, and the 'responder' (\lstinline[language=C]{RESP}), which accepts the connection and can mostly be associated with a server.

We use the following primitives,
\begin{itemize}
    \item \hash(x) is SHA256 of x
    \item $\mathrm{\hash^{n}}$(x) is \hash(x) called iteratively n times
    \item \enc(K,s) is the AES-CTR-128 encryption of s using K as key
\end{itemize}

notation
\begin{itemize}
    \item \lstinline[language=C]{x | y} is the concatenation of x and y
    \item \lstinline[language=C]{UINT32(n)} is the 4 byte value of n in big-endian (network) order
    \item \lstinline[language=C]{SR(n)} is n bytes of strong random data
    \item \lstinline[language=C]{WR(n)} is n bytes of weaker random data
    \item \lstinline[language=C]{"xyz"} is the ASCII characters 'x', 'y', and 'z', not NUL-terminated
    \item \lstinline[language=C]{s[:n]} is the first n bytes of s
    \item \lstinline[language=C]{s[n:]} is the last n bytes of s
\end{itemize}

and constants
\begin{itemize}
    \item \lstinline[language=C]{MAGIC_VALUE = 0x2BF5CA7E}
    \item \lstinline[language=C]{SEED_LENGTH = 16}
    \item \lstinline[language=C]{MAX_PADDING = 8192}
    \item \lstinline[language=C]{HASH_ITERATIONS = 100000}
\end{itemize}

as well as
\begin{itemize}
    \item \lstinline[language=C]{KEYLEN = 16} is the length of the key used by \enc(K,s)
    \item \lstinline[language=C]{IVLEN = 16} is the length of the IV used by \enc(K,s)
    \item \lstinline[language=C]{HASHLEN = 32} is the length of the output of \hash()
    \item \mac(s, x) = \hash(s | x | s)
\end{itemize}

according to \cite{TorGitWebObfs2Specification}. A "byte" is an 8-bit octet and we require that HASHLEN >= KEYLEN + IVLEN.
% ---------------------------------------
\subsection{The key establishment phase}
\label{ss:theKeyestablishmentphase}
% ---------------------------------------
\textit{The key establishment phase} consists of several substeps.
% ---------------------------------------
\subsubsection{The given values}
\label{sss:thegivenvalues}
% ---------------------------------------
Given are the constants \lstinline[language=C]{MAGIC_VALUE}, \lstinline[language=C]{SEED_LENGTH}, \lstinline[language=C]{MAX_PADDING} and \lstinline[language=C]{HASH_ITERATIONS} and the lengths \lstinline[language=C]{KEYLEN}, \lstinline[language=C]{IVLEN} and \lstinline[language=C]{HASHLEN}.
% ---------------------------------------
\subsubsection{Generating initial values}
\label{sss:generatinginitialvalues}
% ---------------------------------------
The 'initiator' (see listing \ref{lst:INIT_seedandpaddingkey}) generate a seed, a padding key and a random PADLEN in range from 0 through \lstinline[language=C]{MAX_PADDING} (inclusive).

\begin{lstlisting}[language=C,  tabsize=4, numbers=left, xleftmargin=5.0ex, basicstyle=\footnotesize, breakatwhitespace=false, breaklines=true, frame=tb, caption=Generate INIT seed and padding key \cite{TorGitWebObfs2Specification}., label=lst:INIT_seedandpaddingkey]
    INIT_SEED = SR(SEED_LENGTH)
    INIT_PAD_KEY = MAC("Initiator obfuscation padding", INIT_SEED)[:KEYLEN]
    PADLEN = R([0:MAX_PADDING])
\end{lstlisting}

The 'responder' (see listing \ref{lst:RESP_seedandpaddingkey}) do it in the same way.
\begin{lstlisting}[language=C,  tabsize=4, numbers=left, xleftmargin=5.0ex, basicstyle=\footnotesize, breakatwhitespace=false, breaklines=true, frame=tb, caption=Generate RESP seed and padding key \cite{TorGitWebObfs2Specification}., label=lst:RESP_seedandpaddingkey]
    RESP_SEED = SR(SEED_LENGTH)
    RESP_PAD_KEY = MAC("Responder obfuscation padding", RESP_SEED)[:KEYLEN]
    PADLEN = R([0:MAX_PADDING])
\end{lstlisting}
% ---------------------------------------
\subsubsection{Init messages}
\label{sss:initmessages}
% ---------------------------------------
After generating the initial values \cite{TorGitWebObfs2Specification}, the initiator (see listing \ref{lst:INIT_initmessage}) sends
\begin{lstlisting}[language=C,  tabsize=4, numbers=left, xleftmargin=5.0ex, basicstyle=\footnotesize, breakatwhitespace=false, breaklines=true, frame=tb, caption=INIT's init message \cite{TorGitWebObfs2Specification}., label=lst:INIT_initmessage]
    INIT_SEED | Enc(INIT_PAD_KEY, UINT32(MAGIC_VALUE) | UINT32(PADLEN) | WR(PADLEN))
\end{lstlisting}
encrypted by the key information given by the \lstinline[language=C]{SEED}, to the responder (see listing \ref{lst:RESP_initmessage}), which do it likewise.
\begin{lstlisting}[language=C,  tabsize=4, numbers=left, xleftmargin=5.0ex, basicstyle=\footnotesize, breakatwhitespace=false, breaklines=true, frame=tb, caption=RESP's init message \cite{TorGitWebObfs2Specification}., label=lst:RESP_initmessage]
    RESP_SEED | Enc(RESP_PAD_KEY, UINT32(MAGIC_VALUE) | UINT32(PADLEN) | WR(PADLEN))
\end{lstlisting}

After receiving the SEED from the other party, each party decrypts the other party's padding key value and the next 8 bytes.

It checks the \lstinline[language=C]{MAGIC_VALUE} and the \lstinline[language=C]{PADLEN} value for conclusiveness and close the connection immediately in case of invalidity, otherwise, it read the remaining \lstinline[language=C]{PADLEN} bytes of padding data and discard them.
% ---------------------------------------
\subsubsection{Final key initialisations}
\label{sss:finalkeyinitialisation}
% ---------------------------------------
If the messages above are valid, the parties derive additional keys (see listing \ref{lst:finalkeyinitialisations}).

\begin{lstlisting}[language=C,  tabsize=4, numbers=left, xleftmargin=5.0ex, basicstyle=\footnotesize, breakatwhitespace=false, breaklines=true, frame=tb, caption=Final key initialisations \cite{TorGitWebObfs2Specification}., label=lst:finalkeyinitialisations]
    INIT_SECRET = MAC("Initiator obfuscated data", INIT_SEED|RESP_SEED)
    RESP_SECRET = MAC("Responder obfuscated data", INIT_SEED|RESP_SEED)

    INIT_KEY = INIT_SECRET[:KEYLEN]
    INIT_IV = INIT_SECRET[KEYLEN:]

    RESP_KEY = RESP_SECRET[:KEYLEN]
    RESP_IV = RESP_SECRET[KEYLEN:]
\end{lstlisting}

The \lstinline[language=C]{INIT_KEY} value keys a stream cipher used to encrypt values from initiator to responder and the stream cipher's IV is \lstinline[language=C]{INIT_IV}. The responder do it in the same way for it's values.
% ================================================================================
\section{Conclusions}
\label{s:conclusions}
% --------------------------------------------------------------------------------
% TODO
% ================================================================================
\appendix
\section{Appendix}
\label{s:appendix}
% --------------------------------------------------------------------------------
% TODO
% ================================================================================
% TODO: entry types and author information, bibliographystyle
%\nocite{*}		% TODO
\bibliography{pt-obfs2}
\bibliographystyle{ACM-Reference-Format}
% ================================================================================
\section*{License}
\label{s:license}
% --------------------------------------------------------------------------------
\begin{center}
	\includegraphics{by-nc-nd.png} \\
	\url{https://creativecommons.org/licenses/by-nc-nd/4.0/}
\end{center}
% ================================================================================
\end{document}
% ================================================================================
